% arara: pdflatex
% arara: biber
% arara: pdflatex
% arara: pdflatex
% --------------------------------------------------------------------------
% the IDXCMDS package
% 
%   create commands for adding formatted index entries
% 
% --------------------------------------------------------------------------
% Clemens Niederberger
% Web:    http://www.mychemistry.eu/forums/forum/idxcmds/
% E-Mail: contact@mychemistry.eu
% --------------------------------------------------------------------------
% Copyright 2012-2015 Clemens Niederberger
% 
% This work may be distributed and/or modified under the
% conditions of the LaTeX Project Public License, either version 1.3
% of this license or (at your option) any later version.
% The latest version of this license is in
%   http://www.latex-project.org/lppl.txt
% and version 1.3 or later is part of all distributions of LaTeX
% version 2005/12/01 or later.
% 
% This work has the LPPL maintenance status `maintained'.
% 
% The Current Maintainer of this work is Clemens Niederberger.
% --------------------------------------------------------------------------
% The idxcmds package consists of the files
%  - idxcmds.sty
%  - idxcmds_en.tex, idxcmds_en.pdf
%  - README
% --------------------------------------------------------------------------
% If you have any ideas, questions, suggestions or bugs to report, please
% feel free to contact me.
% --------------------------------------------------------------------------
\documentclass[load-preamble,add-index,scrartcl={DIV10}]{cnltx-doc}
\usepackage[utf8]{inputenc}
\usepackage{idxcmds}
\setcnltx{
  package  = {idxcmds} ,
  authors  = Clemens Niederberger ,
  email    = contact@mychemistry.eu ,
  url      = http://www.mychemistry.eu/forums/forum/idxcmds/ ,
  abstract = {%
    create commands for adding formatted index entries%
  } ,
  add-cmds = {
    newidxcmd,newsubidxcmd,newsubmainidxcmd
  } ,
  add-silent-cmds = {
    acr,environ,file,hsnbg,KOMAScript,scientist,test,@ifstar,
    cmd,cmd@base,cmd@idx,cmd@nostar,cmd@star,
    indexentry
  } ,
  index-setup = { othercode=\footnotesize,noclearpage } ,
  makeindex-setup = { columns=3, columnsep=1em , intoc , title=Package Index }
}
\usepackage{imakeidx}
\makeindex[name=environments,title=Index of Environments,intoc,columns=2]
\makeindex[name=\jobname-examples,title=Index of Examples,intoc,columns=3]

\usepackage[
backend=biber,
style=cnltx,
sortlocale=en_US,
indexing=cite,
useprefix]{biblatex}
\addbibresource{cnltx.bib}

\defbibheading{bibliography}{\addsec{References}}
\usepackage{csquotes}

\usepackage[biblatex]{embrac}[2012/06/29]
\ChangeEmph{[}[,.02em]{]}[.055em,-.08em]
\ChangeEmph{(}[-.01em,.04em]{)}[.04em,-.05em]

% \setidxcmds{idx-cmd=\index[\jobname-examples]}

\newidxcmd[{\index[\jobname-examples]}]\acr{\textsc{#1}}
\newidxcmd[{\index[environments]}]\environ{\texttt{#1}}[ (Environment)]
\newidxcmd[{\index[\jobname-examples]}]\scientist{\textsf{#1}}
\newsubmainidxcmd[{\index[\jobname-examples]}]\file{\textsf{#1}}
\newsubidxcmd[{\index[\jobname-examples]}]\test{Test}{\textcolor{red}{#1}}
\newsubidxcmd*[{\index[\jobname-examples]}]\hsnbg{\scientist[Heisenberg]{Werner Heisenberg}}{#1}

\begin{document}

\section{Licence and Requirements}\label{sec:license}
\license

\idxcmds\ loads and needs the the packages \pkg{etoolbox}~\cite{pkg:etoolbox},
\pkg{pgfopts}~\cite{pkg:pgfopts} and \pkg{ltxcmds}~\cite{pkg:ltxcmds}.

\section{Motivation}
When working on a larger document and designing and writing the macros for
various bits and pieces I was going to use I found myself repeatedly writing
the same kind of macros again and again that had some kind of semantic
meaning, maybe some markup definitions that also created an index entry and
had a star form for omitting the index entry.  They all had more or less the
following structure:

\begin{sourcecode}
  \makeatletter
  \newcommand*\cmd{\@ifstar\cmd@star\cmd@nostar}
  \newcommand*\cmd@star[1]{\cmd@base{#1}}
  \newcommand*\cmd@nostar[1]{\cmd@base{#1}\cmd@idx{#1}}
  \newcommand*\cmd@base[1]{\textit{#1}}
  \newcommand*\cmd@idx[1]{\index{#1@\cmd@base{#1}}}
  \makeatother
\end{sourcecode}

After having copied and pasted this code for the fourth time I thought: you
should have a command that does this for you.  That was when \cs{newidxcmd}
was born.  This command soon enough got some extensions, \eg, giving
the commands thus defined an optional argument that allowed specifying the
sorting.  It wasn't long before I realized that I might want to use this
\cs{newidxcmd} again in other documents which was when I wrote the first
draft of this package.

I added other commands, \cs{newsubmainidxcmd} and \cs{newsubidxcmd}, which I
didn't (and still don't) really use or need but of which I thought they could
be useful for others, and here we are.

\section{Usage}\label{sec:usage}
\subsection{Available Commands}
\idxcmds\ provides these commands:
\begin{commands}
  \command{newidxcmd}[\oarg{index cs}\marg{cs}\marg{formatting specs}\oarg{append}]
    Defines a command \meta{cs} that formats its argument according to
    \meta{formatting specs} and creates an index entry with \meta{index cs}
    that gets \meta{append} appended.  Refer to the argument of \meta{cs}
    in \meta{formatting specs} with \code{\#1}.
    
    Also defines a command \cs*{\meta{cs}idx} that allows to only create an
    index entry.  See section~\ref{sec:command:usage:newidxcmd} for examples
    and further description of its functionality.  Default for \meta{index cs}
    is \cs{index}.
  \command{newsubidxcmd}[\sarg\oarg{index cs}\marg{cs}\marg{main
    entry}\marg{form. specs}\oarg{append}]
    Defines a command \meta{cs} that formats its argument according to
    \meta{form. specs} and creates an index sub-entry to \meta{main entry}
    with \meta{index cs} that gets \meta{append} appended.  Refer to the
    argument of \meta{cs} in \code{form.\@ specs} with \code{\#1}.
    
    Also defines a command \cs*{\meta{cs}idx} that allows to only create an
    index entry.  See section~\ref{sec:command:usage:newsubidxcmd} for further
    description of its functionality. Default for \meta{index cs} is
    \cs*{index}.
  \command{newsubmainidxcmd}[\oarg{index cs}\marg{cs}\marg{form. specs}\oarg{append}]
    Defines a command \meta{cs} that formats its argument according to
    \meta{form. specs} and creates an index sub-entry to a main entry with
    \meta{index cs} that gets \meta{append} appended.  Refer to the argument
    of \meta{cs} in \meta{form. specs} with \code{\#1}.
    
    Also defines a command \cs*{\meta{cs}idx} that allows to only create an
    index entry.  The main entry is specified at use time. See
    section~\ref{sec:command:usage:newsubmainidxcmd} for examples and further
    description of its functionality. Default for \meta{index cs} is
    \cs*{index}.
\end{commands}

The commands \meta{cs} defined this way are robust but their formatting
argument is not placed in a group.  Keep this in mind when you use
\cs*{bfseries} or something in a definition.  \meta{cs} will be defined using
\pkg{etoolbox}' \cs*{newrobustcmd} which means an error will be thrown if
\meta{cs} is already defined.

Of course these commands cannot cover all possible use cases for index entries
but that is not the intention of this package, anyway.

\subsection{Command Usage}\label{sec:command:usage}
\subsubsection{\cs*{newidxcmd}}\label{sec:command:usage:newidxcmd}
The command \cs{newidxcmd}\Marg{\cs*{foo}}\marg{formatting specs} will define
a new command \cs*{foo} with the following syntax:
\begin{commands}
  \command*{foo}[\sarg\marg{text}]
    format \meta{text} according to specifications, no index entry.
  \command*{foo}[\oarg{sort idx}\marg{text}\oarg{append}]
    format \meta{text} according to specifications, add formatted index
    entry, sorted according to \meta{sort idx} or, if the optional argument
    is left out, sorted according to \meta{text}.  \meta{append} is added to
    the index entry if the trailing optional argument is used (this can be
    useful for instance for getting formatted page numbers).
  \command*{fooidx}[\oarg{sort idx}\marg{text}\oarg{append}]
    add formatted index entry, sorted according to \meta{sort idx} or, if the
    optional argument is left out, sorted according to \meta{text}.
    \meta{append} is added to the index entry if the trailing optional
    argument is used.
\end{commands}

Let's see an example:
\begin{example}
  % in the preamble, probably:
  % \newidxcmd{\acr}{\textsc{#1}}
  % \newidxcmd[{\index[environments]}]{\environ}{\texttt{#1}}[ (Environment)]
  % \newidxcmd{\scientist}{\textsf{#1}}
  \acr{cd}, \acr{id}
  
  \environ{center}, \environ{flushleft}
 
  \scientist*{Albert Einstein}, \scientist[Heisenberg]{Werner Heisenberg}
\end{example}
You will find these examples in the index of examples or the index of
environments, respectively.  The second set of examples shows the purpose of
the first optional argument: if you have several indexes --~like this
documentation has for demonstration purposes~-- you might need to specify the
index command used\footnote{This document uses
  \pkg{imakeidx}~\cite{pkg:imakeidx} for this purpose.}.  If you want to prove
if the example worked: \scientist*{Albert Einstein} should not be found in the
index and \scientist*{Werner Heisenberg} should be sorted under
\emph{Heisenberg}.  Both \environ*{center} and \environ*{flushleft} ahould be
found in the examples index.

Now let's disect the example a bit.  The uses of \cs*{acr}\Marg{cd},
\cs*{scientist}\sarg\Marg{Albert Einstein} and \cs*{environ}\Marg{center} will
essentially expand to
\begin{sourcecode}
  % \acr{cd} =>
  \textsc{cd}\index{cd@\textsc{cd}}
  % and in \jobname.idx =>
  \indexentry{cd@\textsc  {cd}}{3}
 
  % \scientist*{Albert Einstein} =>
  \textsf{Albert Einstein}
 
  % \environ{center} =>
  \texttt{center}\index[environments]{center@\texttt{center} (Environment)}
  % and in examples.idx =>
  \indexentry{center@\texttt  {center} (Environment)}{3}
\end{sourcecode}

\subsubsection{\cs*{newsubidxcmd}}\label{sec:command:usage:newsubidxcmd}
The command \cs{newsubidxcmd}\sarg\Marg{\cs*{foo}}\marg{main entry}\marg{form.
  specs} will define a new command \cs*{foo} with the same syntax as
\cs{newidxcmd} does.  However, \cs{newsubidxcmd} has an additional argument
that specifies the main index entry this group of sub entries belongs to.  For
the unstarred variant this argument can be some arbitrary main entry.  For the
starred variant it demands a command plus argument defined by \cs{newidxcmd}
as argument.

\begin{example}
  % preamble:
  % \newsubidxcmd{\test}{Test}{\textcolor{red}{#1}}
  % \newsubidxcmd*{\hsnbg}{\scientist[Heisenberg]{Werner Heisenberg}}{#1}
  \scientist[Heisenberg]{Werner Heisenberg} was born in
  \hsnbg[Wuerzburg]{W\"urzburg (Germany)}. He worked as a
  professor in \hsnbg{Leipzig (Germany)}. And this is a
  \test{test}.
\end{example}

\subsubsection{\cs*{newsubmainidxcmd}}\label{sec:command:usage:newsubmainidxcmd}
The command \cs{newsubmainidxcmd}\Marg{\cs*{foo}}\marg{form. specs} will define
a new command \cs*{foo} similar to \cs{newsubidxcmd} but where the main
index entry is specified for every use case in the running text.  \cs*{foo}
will have the following syntax:
\begin{commands}
  \command*{foo}[\sarg\marg{text}]
    format \meta{text} according to specifications, no index entry.
  \command*{foo}[\oarg{sort idx}\marg{text}\marg{main entry}\oarg{append}]
    format \meta{text} according to specifications, add formatted index
    sub-entry to the main index entry \meta{main entry}, sorted according to
    \meta{sort idx} or, if the optional argument is left out, sorted
    according to \meta{text}.  \meta{append} is added to the index entry if
    the trailing optional argument is used (this can be useful for instance
    for getting formatted page numbers).
  \command*{fooidx}[\oarg{sort idx}\marg{text}\marg{main entry}\oarg{append}]
    add formatted index sub-entry to the main index entry \meta{main entry},
    sorted according to \meta{sort idx} or, if the optional argument is left
    out, sorted according to \meta{text}.  \meta{append} is added to the
    index entry if the trailing optional argument is used.
\end{commands}

\begin{example}
  % in the preamble, probably:
  % \newsubmainidxcmd{\file}{\textsf{#1}}
  \file{article}{classes} is a standard \LaTeX{} class.
  \file{scrartcl}{KOMA-Script@\KOMAScript} is part of the \KOMAScript{} bundle.
  \file*{test} is a dummy.
\end{example}

\subsection{Options}
\idxcmds\ has the following options -- either setable as package option or with
the setup command \cs{setidxcmds}\marg{options} which takes a comma separated
list of options:
\begin{options}
  \keyval{sort-sep}{symbol}\Default{@}
    set makeindex symbol to separate the index into sorting and typesetting
    part as specified in the index style file.
  \keyval{sub-sep}{symbol}\Default{!}
    set makeindex symbol to add a sub entry as specified in the index style
    file.
  \keyval{idx-cmd}{cs}\Default{\cs*{index}}
    \sinceversion{0.2b}The default indexing command that is used by \idxcmds.
\end{options}

% \end{document}

\printbibliography

% \section{Implementation}
% In the following code the lines 1--32 have been omitted.  They only repeat the
% license statement which has already been mentioned in
% section~\ref{sec:license}.

% \implementation[linerange={33-1000},firstnumber=33]


\printindex[\jobname-examples]

\printindex[environments]

% \indexprologue{\noindent
%   This index has some peculiar entries in addition to the ones you would
%   normally expect.  But actually they are the corresponding entries to the
%   examples shown in section~\ref{sec:usage}.%
% }
% \printindex

\end{document}

% arara: pdflatex
% arara: biber
% arara: pdflatex
% arara: pdflatex
% --------------------------------------------------------------------------
% the IDXCMDS package
% 
%   create commands for adding formatted index entries
% 
% --------------------------------------------------------------------------
% Clemens Niederberger
% Web:    https://bitbucket.org/cgnieder/idxcmds/
% E-Mail: contact@mychemistry.eu
% --------------------------------------------------------------------------
% Copyright 2012 Clemens Niederberger
% 
% This work may be distributed and/or modified under the
% conditions of the LaTeX Project Public License, either version 1.3
% of this license or (at your option) any later version.
% The latest version of this license is in
%   http://www.latex-project.org/lppl.txt
% and version 1.3 or later is part of all distributions of LaTeX
% version 2005/12/01 or later.
% 
% This work has the LPPL maintenance status `maintained'.
% 
% The Current Maintainer of this work is Clemens Niederberger.
% --------------------------------------------------------------------------
% The idxcmds package consists of the files
%  - idxcmds.sty
%  - idxcmds_en.tex, idxcmds_en.pdf
%  - README
% --------------------------------------------------------------------------
% If you have any ideas, questions, suggestions or bugs to report, please
% feel free to contact me.
% --------------------------------------------------------------------------
% in order to compile this documentation you need the document class
% `cnpkgdoc' which you can get here:
%   https://bitbucket.org/cgnieder/cnpkgdoc/
%
\PassOptionsToPackage{supstfm=libertinesups}{superiors}
\documentclass{cnpkgdoc}
\docsetup{
  pkg = idxcmds ,
  subtitle = create commands for adding formatted index entries,
  language = en ,
  code-box = {
    backgroundcolor  = gray!5!white ,
    skipbelow        = .6\baselineskip plus .5ex minus .5ex ,
    skipabove        = .6\baselineskip plus .5ex minus .5ex ,
    roundcorner      = 3pt ,
  } ,
  gobble   = 1
}

\addcmds{acr,environ,file,hsnbg,KOMAScript,name,test}
\addcmds{
  @ifstar,
  cmd,cmd@base,cmd@idx,cmd@nostar,cmd@star
}

\cnpkgusecolorscheme{friendly}
\usepackage{libertinehologopatch}
\renewcommand*\othersectionlevelsformat[3]{%
  \textcolor{main}{#3\autodot}\enskip}
\renewcommand*\partformat{%
  \textcolor{main}{\partname~\thepart\autodot}}
\usepackage{fnpct}
\pagestyle{headings}

\usepackage{imakeidx}
\usepackage{filecontents}
\begin{filecontents*}{\jobname.ist}
 heading_prefix "{\\bfseries "
 heading_suffix "\\hfil}\\nopagebreak\n"
 headings_flag  1
 delim_0 "\\dotfill\\hyperpage{"
 delim_1 "\\dotfill\\hyperpage{"
 delim_2 "\\dotfill\\hyperpage{"
 delim_r "}\\textendash\\hyperpage{"
 delim_t "}"
 suffix_2p "\\nohyperpage{\\,f.}"
 suffix_3p "\\nohyperpage{\\,ff.}"
\end{filecontents*}
\indexsetup{othercode=\footnotesize,noclearpage}
\makeindex[options={-s \jobname.ist},intoc,columns=3]
\makeindex[name=examples,title=Example Index,intoc,columns=3]

\usepackage[backend=biber,style=alphabetic]{biblatex}
\addbibresource{\jobname.bib}
\begin{filecontents}{\jobname.bib}
@package{pkg:etoolbox,
  title   = {\paket*{etoolbox}},
  author  = {Philipp Lehman},
  date    = {2011-01-21},
  version = {2.1},
  url     = {http://mirror.ctan.org/macros/latex/contrib/etoolbox}
}
@package{pkg:imakeidx,
  title   = {\paket*{imakeidx}},
  author  = {Claudio Beccari and Enrico Gregorio},
  date    = {2013-03-26},
  version = {1.3},
  url     = {http://mirror.ctan.org/macros/latex/contrib/imakeidx}
}
@package{pkg:pgfopts,
  title   = {\paket*{pgfopts}},
  author  = {Joseph Wright},
  date    = {2011-06-02},
  version = {2.1},
  url     = {http://mirror.ctan.org/macros/latex/contrib/pgfopts}
}
\end{filecontents}

\newcommand*\Default[1]{%
  \hfill\llap
    {%
      \ifblank{#1}%
        {(initially~empty)}%
        {Default:~\code{#1}}%
    }%
  \newline
}

\newidxcmd\acr{\textsc{#1}}
\newidxcmd[{\index[examples]}]\environ{\texttt{#1}}[ (Environment)]
\newidxcmd\name{\textsf{#1}}
\newsubmainidxcmd\file{\textsf{#1}}
\newsubidxcmd\test{Test}{\textcolor{red}{#1}}
\newsubidxcmd*\hsnbg{\name[Heisenberg]{Werner Heisenberg}}{#1}

\begin{document}

\section{Licence and Requirements}\label{sec:license}\secidx{Licence}
Permission is granted to copy, distribute and/or modify this software under the
terms of the \LaTeX{} Project Public License, version 1.3 or later
(\url{http://www.latex-project.org/lppl.txt}). The package has the status
``maintained.''

\idxcmds loads and needs the the packages \paket{etoolbox}~\cite{pkg:etoolbox}
and \paket{pgfopts}~\cite{pkg:pgfopts}.
\secidx*{Licence}

\section{Motivation}\secidx{Motivation}
When working on a larger document and designing and writing the macros for various
bits and pieces I was going to use I found myself repeatedly writing the same
kind of macros again and again that had some kind og semantic meaning, maybe some
markup definitions that also created an index entry and had a star form for
omitting the index entry. They all had more or less the following structure:

\begin{beispiel}[code only]
 \makeatletter
 \newcommand*\cmd{\@ifstar\cmd@star\cmd@nostar}
 \newcommand*\cmd@star[1]{\cmd@base{#1}}
 \newcommand*\cmd@nostar[1]{\cmd@base{#1}\cmd@idx{#1}}
 \newcommand*\cmd@base[1]{\textit{#1}}
 \newcommand*\cmd@idx[1]{\index{#1@\cmd@base{#1}}}
 \makeatother
\end{beispiel}

After having copied and pasted this code for the fourth time I thought: you
should have a command that does this for you. That was when \cmd{newidxcmd}
was born. This command soon enough got some extensions, \emph{e.g.}, giving the
commands thus defined an optional argument that allowed specifying the sorting.
It wasn't long before I realized that I might want to use this \cmd{newidxcmd}
again in other documents which was when I wrote the first draft of this package.

I added other commands, \cmd{newsubmainidxcmd} and \cmd{newsubidxcmd}, which I
didn't (and still don't) really use or need but of which I thought they could be
useful for others, and here we are.
\secidx*{Motivation}

\section{Usage}\label{sec:usage}\secidx{Usage}
\subsection{Available Commands}\secidx[Available Commands]{Usage}
\idxcmds provides these commands:
\begin{beschreibung}
 \Befehl{newidxcmd}[<index cs>]{\cmd*{cmd}}\ma{<formatting specs>}\oa{<app to idx entry>}\newline
   Defines a command \cmd*{cmd} that formats its argument according to
   \code{<formatting specs>} and creates an index entry with \code{<index cs>}
   that gets \code{<app to idx entry>} appended. Also defines a command \cmd*{cmdidx}
   that allows to only create an index entry. See section~\ref{ssec:command:usage:newidxcmd}
   for examples and further description of its functionality. Default for
   \code{<index cs>} is \cmd{index}.
 \Befehl{newsubidxcmd}*[<index cs>]{\cmd*{cmd}}\ma{<main entry>}\ma{<form.\@ specs>}\oa{<app to i.\@ e.>}\newline
   Defines a command \cmd*{cmd} that formats its argument according to
   \code{<form.\@ specs>} and creates an index sub-entry to \code{<main entry>}
   with \code{<index cs>} that gets \code{<app to i.\@ e.>} appended. Also
   defines a command \cmd*{cmdidx} that allows to only create an index entry.
   See section~\ref{ssec:command:usage:newsubidxcmd} for further description of
   its functionality. Default for \code{<index cs>} is \cmd{index}.
 \Befehl{newsubmainidxcmd}[<index cs>]{\cmd*{cmd}}\ma{<form.\@ specs>}\oa{<app to i.\@ e.>}\newline
   Defines a command \cmd*{cmd} that formats its argument according to
   \code{<form.\@ specs>} and creates an index sub-entry to a main entry with
   \code{<index cs>} that gets \code{<app to i.\@ e.>} appended. Also defines a
   command \cmd*{cmdidx} that allows to only create an index entry. The main entry
   is specified at use time. See section~\ref{ssec:command:usage:newsubmainidxcmd}
   for examples and further description of its functionality. Default for
   \code{<index cs>} is \cmd{index}.
\end{beschreibung}

The commands \cmd*{cmd} defined this way are robust but their formatting argument
is not placed in a group. Keep this in mind when you use \cmd*{bfseries} or
something in a definition.

Of course these commands cannot cover all possible use cases for index entries
but that is not the intention of this package, anyway.

\subsection{Command Usage}\label{ssec:command:usage}\secidx[Command Usage]{Usage}
\subsubsection{\cmd*{newidxcmd}}\label{ssec:command:usage:newidxcmd}
The command \cmd{newidxcmd}{\cmd*{cmd}}\ma{<formatting specs>} will define a new
command \cmd*{cmd} with the following syntax:
\begin{beschreibung}
 \Befehl{cmd}*{<text>}\newline
   format \code{<text>} according to specifications, no index entry.
 \Befehl{cmd}{<text>}\newline
   format \code{<text>} according to specifications, add formatted index entry,
   sorted according to \code{<text>}.
 \Befehl{cmd}[<srt idx>]{<text>}\newline
   format \code{<text>} according to specifications, add formatted index entry,
   sorted according to \code{<srt idx>}.
 \Befehl{cmdidx}[<srt idx>]{<text>}\newline
   add formatted index entry, sorted according to \code{<srt idx>}.
\end{beschreibung}

Let's see an example:
\begin{beispiel}
 % in the preamble, probably:
 % \newidxcmd{\acr}{\textsc{#1}}
 % \newidxcmd[{\index[examples]}]{\environ}{\texttt{#1}}[ (Environment)]
 % \newidxcmd{\name}{\textsf{#1}}
 \acr{cd}, \acr{id}
 
 \environ{center}, \environ{flushleft}

 \name*{Albert Einstein}, \name[Heisenberg]{Werner Heisenberg}
\end{beispiel}
You will find these examples in the index or the examples index, respectively.
The second set of examples shows the purpose of the first optional argument: if
you have several indexes --~like this documentation has for demonstration
purposes~-- you might need to specify the index command used\footnote{This
document uses \paket*{imakeidx}~\cite{pkg:imakeidx} for this purpose.}. If you
want to prove if the example worked: \name*{Albert Einstein} should not be
found in the index and \name*{Werner Heisenberg} should be sorted under
\emph{Heisenberg}. Both \environ*{center} and \environ*{flushleft} ahould be
found in the examples index.

Now let's disect the example a bit. The uses of \cmd*{acr}{cd},
\cmd*{name}*{Albert Einstein} and \cmd*{environ}{center} will essentially expand
to
\begin{beispiel}[code only]
 % \acr{cd} =>
 \textsc{cd}\index{cd@\textsc{cd}}
 % \name*{Albert Einstein} =>
 \textsf{Albert Einstein}
 % \environ{center} =>
 \texttt{center}\index[examples]{center@\texttt{center} (Environment)}
\end{beispiel}

\subsubsection{\cmd*{newsubidxcmd}}\label{ssec:command:usage:newsubidxcmd}
The command \cmd{newsubidxcmd}*{\cmd*{cmd}}\ma{<main entry>}\ma{<form.\@ specs>}
will define a new command \cmd*{cmd} with the same syntax as \cmd{newidxcmd} does.
However, \cmd{newsubidxcmd} has an additional argument that specifies the main
index entry this group of sub entries belongs to. For the unstarred variant this
argument can be some arbitrary main entry. For the starred variant it demands a
command plus argument defined by \cmd{newidxcmd} as argument.

\begin{beispiel}
 % preamble:
 % \newsubidxcmd{\test}{Test}{\textcolor{red}{#1}}
 % \newsubidxcmd*{\hsnbg}{\name[Heisenberg]{Werner Heisenberg}}{#1}
 \name[Heisenberg]{Werner Heisenberg} was born in \hsnbg{W\"urzburg
 (Germany)}. He worked as a professor in \hsnbg{Leipzig (Germany)}.
 And this is a \test{test}.
\end{beispiel}

\subsubsection{\cmd*{newsubmainidxcmd}}\label{ssec:command:usage:newsubmainidxcmd}
The command \cmd{newsubmainidxcmd}{\cmd*{cmd}}\ma{<form.\@ specs>} will define a
new command \cmd*{cmd} similar to \cmd{newsubidxcmd} but where the main index
entry is specified for every use case in the running text. \cmd*{cmd} will have
the following syntax:
\begin{beschreibung}
 \Befehl{cmd}*{<text>}\newline
   format \code{<text>} according to specifications, no index entry.
 \Befehl{cmd}{<text>}\ma{<main entry>}\newline
   format \code{<text>} according to specifications, add formatted index sub-entry
   to the main index entry \code{<main entry>}, sorted according to \code{<text>}.
 \Befehl{cmd}[<srt idx>]{<text>}\ma{<main entry>}\newline
   format \code{<text>} according to specifications, add formatted index sub-entry
   to the main index entry \code{<main entry>}, sorted according to \code{<srt idx>}.
 \Befehl{cmdidx}[<srt idx>]{<text>}\ma{<main entry>}\newline
   add formatted index sub-entry to the main index entry \code{<main entry>},
   sorted according to \code{<srt idx>}.
\end{beschreibung}

\begin{beispiel}
 % in the preamble, probably:
 % \newsubmainidxcmd{\file}{\textsf{#1}}
 \file{article}{classes} is a standard \LaTeX{} class.
 \file{scrartcl}{KOMA-Script@\KOMAScript} is part of the \KOMAScript{} bundle.
 \file*{test} is a dummy.
\end{beispiel}

\subsection{Options}\secidx[Options]{Usage}
\idxcmds has these option:
\begin{beschreibung}
 \Option{sort-sep}{<symbol>}\Default{@}
   set makeindex symbol to separate the index into sorting and typesetting part
   as specified in the index style file.
 \Option{sub-sep}{<symbol>}\Default{!}
   set makeindex symbol to add a sub entry as specified in the index style file.
\end{beschreibung}
\secidx*{Usage}

\printbibliography

\section{Implementation}\secidx{Implementation}
In the following code the lines 1--32 have been omitted. They only repeat the
license statement which has already been mentioned in section~\ref{sec:license}.

\implementation[linerange={33-1000},firstnumber=33]
\secidx*{Implementation}

% \begingroup
\indexprologue{\noindent
  This index has some peculiar entries in addition to the ones you
  would normally expect. But actually they are the corresponding entries to the
  examples shown in section~\ref{sec:usage}.%
}
\printindex
% \def\clearpage{\vskip\baselineskip}
\clearpage
\printindex[examples]
% \endgroup

\end{document}
\documentclass{cnpkgdoc}
\docsetup{
  pkg = idxcmds ,
  language = en ,
  code-box = {
    backgroundcolor  = gray!7!white ,
    skipbelow        = .6\baselineskip plus .5ex minus .5ex ,
    skipabove        = .6\baselineskip plus .5ex minus .5ex ,
    roundcorner      = 3pt ,
  } ,
  gobble   = 1
}
\addcmds{acr,environ,file,hsnbg,KOMAScript,name,test}

\cnpkgusecolorscheme{friendly}
\usepackage{fnpct}

\usepackage{imakeidx}
\makeindex[intoc]
\makeindex[name=examples,title=Example Index,intoc]

\newcommand*\Default[1]{%
  \hfill\llap
    {%
      \ifblank{#1}%
        {(initially~empty)}%
        {Default:~\code{#1}}%
    }%
  \newline
}

\newidxcmd\acr{\textsc{#1}}
\newidxcmd[{\index[examples]}]\environ{\texttt{#1}}[ (Environment)]
\newidxcmd\name{\textsf{#1}}
\newsubmainidxcmd\file{\textsf{#1}}
\newsubidxcmd\test{Test}{\textcolor{red}{#1}}
\newsubidxcmd*\hsnbg{\name[Heisenberg]{Werner Heisenberg}}{#1}

\begin{document}

\section{Licence and Requirements}
Permission is granted to copy, distribute and/or modify this software under the
terms of the \LaTeX{} Project Public License, version 1.3 or later
(\url{http://www.latex-project.org/lppl.txt}). The package has the status
``maintained.''

\idxcmds loads and needs the the package \paket{etoolbox}.

\section{Usage}
\subsection{Available Commands}
\idxcmds provides these commands:
\begin{beschreibung}
 \Befehl{newidxcmd}[<index cs>]{\cmd{cmd}}\ma{<formatting specs>}\oa{<app to idx entry>}\newline
   Defines a command \cmd{cmd}. See next section for a description. Default for
   \code{<index cs>} is \cmd{index}.
 \Befehl{newsubmainidxcmd}[<index cs>]{\cmd{cmd}}\ma{<formatting specs>}\oa{<app to i. e.>}\newline
   Defines a command \cmd{cmd}. See next section for a description. Default for
   \code{<index cs>} is \cmd{index}.
 \Befehl{newsubidxcmd}*[<index cs>]{\cmd{cmd}}\ma{<main entry>}\ma{<formatting specs>}\oa{<app to i. e.>}\newline
   Defines a command \cmd{cmd}. See next section for a description. Default for
   \code{<index cs>} is \cmd{index}.
\end{beschreibung}

\subsection{Command Usage}
\cmd{newidxcmd} will define a new command with the following syntax:
\begin{beschreibung}
 \befehl{cmd}*{<text>} format \code{<text>} according to specifications, no
   index entry.
 \befehl{cmd}{<text>} format \code{<text>} according to specifications, add
   formatted index entry, sorted according to \code{<text>}.
 \befehl{cmd}[<srt idx>]{<text>} format \code{<text>} according to specifications,
   add formatted index entry, sorted according to \code{<srt idx>}.
\end{beschreibung}

Let's see an example:
\begin{beispiel}
 % preamble:
 % \newidxcmd\acr{\textsc{#1}}
 % \newidxcmd[{\index[examples]}]\environ{\texttt{#1}}[ (Environment)]
 % \newidxcmd\name{\textsf{#1}}
 \acr{cd}, \acr{id}
 
 \environ{center}, \environ{flushleft}

 \name*{Albert Einstein}, \name[Heisenberg]{Werner Heisenberg}
\end{beispiel}
You will find these examples in the index or the examples index, resp. The second
set of examples shows the purpose of the first optional argument: if you have
several indexes -- like this documentation has for demonstration purposes --
you might need to specify the index command used. This document uses \paket{imakeidx}
for this purpose.

\cmd{newsubmainidxcmd} will define a new command with the following syntax:
\begin{beschreibung}
 \befehl{cmd}*{<text>} format \code{<text>} according to specifications, no
   index entry.
 \Befehl{cmd}{<text>}\ma{<main entry>} \cnpkgdocarrow{} format \code{<text>}
   according to specifications, add formatted index entry, sorted according to
   \code{<text>}.
 \Befehl{cmd}[<srt idx>]{<text>}\ma{<main entry>} \cnpkgdocarrow{} format
   \code{<text>} according to specifications, add formatted index entry, sorted
   according to \code{<srt idx>}.
\end{beschreibung}

\begin{beispiel}
 % preamble
 % \newsubmainidxcmd\file{\textsf{#1}}
 \file{article}{classes} is a standard \LaTeX{} class.
 \file{scrartcl}{KOMA-Script@\KOMAScript} is part of the \KOMAScript{} bundle.
 \file*{test} is an dummy.
\end{beispiel}

\cmd{newsubidxcmd} will define a new command with the same syntax as \cmd{newidxcmd}
does. However, \cmd{newsubidxcmd} has an additional argument that specifies the
main entry this group of sub entries belongs to. In the unstarred variant this
argument can be some arbitrary main entry. In the starred variant it demands
a command plus argument defined by \cmd{newidxcmd} as argument.

\begin{beispiel}
 % preamble:
 % \newsubidxcmd\test{Test}{\textcolor{red}{#1}}
 % \newsubidxcmd*\hsnbg{\name[Heisenberg]{Werner Heisenberg}}{#1}
 \name[Heisenberg]{Werner Heisenberg} was born in \hsnbg{W\"urzburg, Germany}.
 And this is a \test{test}.
\end{beispiel}

The commands defined this way are robust but their formatting is not grouped. So
keep this in mind when you use \cmd*{bfseries} or something in a definition.

Of course these commands cannot cover all possible use cases but this is not the
intention, anyway.

\subsection{Options}
\idxcmds has these option:
\begin{beschreibung}
 \Option{sortsep}{<symbol>}\Default{@}
   set makeindex symbol to separate the index into sorting and typesetting part
   as specified in the index style file.
 \Option{subsep}{<symbol>}\Default{!}
   set makeindex symbol to add a sub entry as specified in the index style file.
\end{beschreibung}


\begingroup
\printindex
\def\clearpage{\vskip\baselineskip}
\printindex[examples]
\endgroup

\end{document}